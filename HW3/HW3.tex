\documentclass[12pt, letterpaper]{article}
\usepackage[utf8]{inputenc}
\usepackage{amsmath, amsfonts, amssymb}
\usepackage{geometry}
\usepackage{graphicx}
\usepackage{fancyhdr}
\usepackage{pgfplots}
\usepackage{float}
\usepackage{pgfplotstable}
\usepackage{booktabs} 
\pgfplotsset{compat=1.18}

% Setup for homework style
\geometry{margin=1in}
\pagestyle{fancy}
\fancyhf{}
\lhead{APPM 2360: Differential Equations}
\rhead{Homework 3}
\cfoot{\thepage}

\title{Homework 3}
\author{Zachariah Galdston}
\date{9/12/2024}

\begin{document}

\maketitle

\section*{Problem 1}

\textbf{Problem Statement:} Determine weather Picard's Theorem can be used to show the existence of a unique solution in an open interval containing t = 0

\textbf{Solution:} Picards theorem sates that if $f(t,y)$ is continuous in some region $R$ defined by 
\[ \{(t,y) \,|\, a < t < b, c < y < d\} \] 

and $(t_0, y_0) \in \mathbb{R}$. Then there exists a postivie number $h$ such that the IVP
\[ y' = f(t,y), \quad y(t_0) = y_0\]
Has a solution in an open interval containing $(t_0 - h, t_0 + h)$. The solution is unique if $\frac{\partial f}{\partial y}$ is continuous in $R$

\subsection*{(a) $y' = ty^{\frac{4}{3}}, \quad y(0) = 0$}
\textbf{Problem Statement:} Determine weather Picard's Theorem can be used to show the existence of a unique solution in an open interval containing t = 0

\textbf{Solution:} We will start by determining the continutiy of the function $f(t) = ty^{\frac{4}{3}}$. We will start by finding the partial derivative of $f$ with respect to $y$ \\

$f(t) = ty^{\frac{4}{3}}$ is continuous for all $t$ and $y$ in the region $R$ defined by $a \leq t \leq b$ and $|y - y_0| \leq M$ for some constant $M$. Therefore, Picard's Theorem can be used to show the existence of a  solution in an open interval containing $t = 0$. \\

Next we will determine the uniqueness of the solution. We will start by finding the partial derivative of $f$ with respect to $y$

\begin{align*}
\frac{\partial f}{\partial y} &= \frac{4}{3}ty^{\frac{1}{3}}
\end{align*}

Since $\frac{\partial f}{\partial y}$ is continuous for all $t$ and $y$ in the region $R$ defined by $a \leq t \leq b$ and $|y - y_0| \leq M$ for some constant $M$, Picard's Theorem can be used to show the existence of a unique solution in an open interval containing $t = 0$.

\subsection*{(b) $y' = ty^{1/3}, \quad y(0) = 0$}
\textbf{Problem Statement:} Determine weather Picard's Theorem can be used to show the existence of a unique solution in an open interval containing t = 0

\textbf{Solution:} We will start by determining the continutiy of the function $f(t) = ty^{\frac{1}{3}}$ 

$f(t) = ty^{\frac{1}{3}}$ is continuous for all $t$ and $y$ in the region $R$ Therefore, Picard's Theorem can be used to show the existence of a  solution in an open interval containing $t = 0$. \\

Next we will determine the uniqueness of the solution. We will start by finding the partial derivative of $f$ with respect to $y$ \\
\begin{align*}
\frac{\partial f}{\partial y} &= \frac{1}{3}ty^{-\frac{2}{3}}
\end{align*}

Since $\frac{\partial f}{\partial y}$ is not continuous for all $t$ and $y$ in the region $R$ defined by $a \leq t \leq b$ and $|y - y_0| \leq M$ for some constant $M$, Picard's Theorem cannot be used to show the existence of a unique solution in an open interval containing $t = 0$.

\subsection*{(c) $y' = ty^{\frac{1}{3}}, \quad y(0) = 1$}
\textbf{Problem Statement:} Determine weather Picard's Theorem can be used to show the existence of a unique solution in an open interval containing t = 0

\textbf{Solution:} We will start by determining the continutiy of the function $f(t) = ty^{\frac{1}{3}}$

$f(t) = ty^{\frac{1}{3}}$ is continuous for all $t$ and $y$ in the region $R$ defined by $a \leq t \leq b$ and $|y - y_0| \leq M$ for some constant $M$. Therefore, Picard's Theorem can be used to show the existence of a  solution in an open interval containing $t = 0$. \\
Next we will determine the uniqueness of the solution. We will start by finding the partial derivative of $f$ with respect to $y$ \\
\begin{align*}
\frac{\partial f}{\partial y} &= \frac{1}{3}ty^{-\frac{2}{3}}
\end{align*}

Since $\frac{\partial f}{\partial y}$ is continuous for all $t$ and $y$ in the region $R$ defined by $a \leq t \leq b$ and $|y - y_0| \leq M$ for some constant $M$, Picard's Theorem can be used to show the existence of a unique solution in an open interval containing $t = 0$.

\section*{Problem 2}

\textbf{Problem Statement:} Find the order, linearity, homogeneity, and the variablity of the coefficients of the following:

%I want to make a table for this problem, with the diff eqs a-e on the left and then the solutions on the right
\begin{table}[H]
    \centering
    \begin{tabular}{|c|c|c|c|c|c|}
        \hline
        \textbf{Problem} & \textbf{Diff Eq} & \textbf{Order} & \textbf{Linearity} & \textbf{Homogeneity} & \textbf{coefficients} \\
        \hline
        (a) & $y'' + 2y' + y = 0$ & 2 & Linear & Homogeneous & Constant \\
        \hline
        (b) & $\ddot x + 2\dot x + tx = \sin(t)$ & 2 & Linear & Non-Homogeneous & Variable \\
        \hline
        (c) & $\cos(y') + ty = 0$ & 1 & Non-Linear & Homogeneous & Variable \\
        \hline
        (d) & $y'' + ey' + \pi y = 0$ & 2 & Linear & Homogeneous & Constant \\
        \hline
        (e) & $y' + \frac{1}{1+t^2}y = 7$ & 1 & Linear & Non-Homogeneous & Variable \\
        \hline
    \end{tabular}
\end{table}

\section*{Problem 3}

\textbf{Problem Statement:} Which of the following operators are linear?

\textbf{Solution:} An operator is linear if it satisfies the following properties: 
\begin{align*}
    L(k\vec{\textbf{u}}) = kL(\vec{\textbf{u}}), \quad k \in \mathbb{R} \\
    L(\vec{\textbf{u}} + \vec{\textbf{w}}) = L(\vec{\textbf{u}}) + L(\vec{\textbf{w}})
\end{align*}

\subsection*{(a) $L(\vec{\textbf{y}}) = y' + 2ty$}
\textbf{Solution:} We will start by checking the first property
\begin{align*}
    L(k\vec{\textbf{u}}) &= kL(\vec{\textbf{u}}) \\
    L(ky) &= k(y' + 2ty) \\
    ky' + 2kty &= ky' + 2kty \\
    ky' + 2kty &= ky' + 2kty
\end{align*}

Now we will check the second property

\begin{align*}
    L(\vec{\textbf{u}} + \vec{\textbf{w}}) &= L(\vec{\textbf{u}}) + L(\vec{\textbf{w}}) \\
    L(y_1 + y_2) &= L(y_1) + L(y_2) \\
    y_1' + 2ty_1 + y_2' + 2ty_2 &= y_1' + 2ty_1 + y_2' + 2ty_2
\end{align*}

Since both properties are satisfied, the operator $L(\vec{\textbf{y}}) = y' + 2ty$ is linear

\subsection*{(b) $L(\vec{\textbf{y}}) = y'' + (1-y^2) + y$}
\textbf{Solution:} We will start by checking the first property
\begin{align*}
    L(k\vec{\textbf{u}}) &= kL(\vec{\textbf{u}}) \\
    L(ky) &= k(y'' + (1-y^2) + y) \\
    ky'' + (1-(ky)^2) + ky &\neq ky'' + k(1-y^2) + ky \\
\end{align*}

This does not satisfy the first property, so the operator $L(\vec{\textbf{y}}) = y'' + (1-y^2) + y$ is not linear. We will check the second property to be thorough.

\begin{align*}
    L(\vec{\textbf{u}} + \vec{\textbf{w}}) &= L(\vec{\textbf{u}}) + L(\vec{\textbf{w}}) \\
    L(y_1 + y_2) &= L(y_1) + L(y_2) \\
    y_1'' + (1-y_1^2) + y_1 + y_2'' + (1-y_2^2) + y_2 &\neq y_1'' + (1-y_1^2) + y_1 + y_2'' + (1-y_2^2) + y_2
\end{align*}

Since the second property is not satisfied, the operator $L(\vec{\textbf{y}}) = y'' + (1-y^2) + y$ is not linear.

\section*{Problem 4}
\textbf{Problem Statement:} Show that if $y_1(t)$ and $y_2(t)$ are solutions of $y' + p(t)y = 0$, then so are $y_1(t) + y_2(t)$ and $cy_1(t)$ for any constant $c$.

\textbf{Solution:} We will start by assuming that $y_1(t)$ and $y_2(t)$ are solutions of $y' + p(t)y = 0$ and follow the properties of linear homogenoues ODEs. Thus,
\begin{align*}
    y_1' + p(t)y_1 &= 0 \\
    y_2' + p(t)y_2 &= 0
\end{align*}

Using the first property, we will show that $y_1(t) + y_2(t)$ is a solution of $y' + p(t)y = 0$
\begin{align*}
    (y_1 + y_2)' + p(t)(y_1 + y_2) &= 0 \\
    y_1' + y_2' + p(t)y_1 + p(t)y_2 &= 0 \\
    y_1' + p(t)y_1 + y_2' + p(t)y_2 &= 0 + 0 = 0 
\end{align*}

This proves that $y_1(t) + y_2(t)$ is a solution of $y' + p(t)y = 0$. 

Next we will show that $cy_1(t)$ is a solution of $y' + p(t)y = 0$
\begin{align*}
    (cy_1)' + p(t)(cy_1) &= 0 \\
    cy_1' + cp(t)y_1 &= 0 \\
    c(y_1' + p(t)y_1) &= c \cdot 0 = 0 
\end{align*}

This proves that $cy_1(t)$ is a solution of $y' + p(t)y = 0$.

\section*{Problem 5}
\textbf{Problem Statement:} Verify that the given functions $y_1(t)$ and $y_2(t)$ are solutions of the given differential equation. Then show that $c_1y_1(t) + c_2y_2(t)$ is also a solution for any real numbers $c_1$ and $c_2$.
\subsection*{(a) $y'' - y' + 6y = 0; \quad y_1(t) = e^{3t}, \quad y_2(t) = e^{-2t}$}
\textbf{Solution:} We will start by verifying that $y_1(t)$ and $y_2(t)$ are solutions of the given differential equation.
\begin{align*}
    y_1(t) &= e^{3t} \\
    y_1' &= 3e^{3t} \\
    y_1'' &= 9e^{3t} \\
    9e^{3t} - 3e^{3t} - 6e^{3t} &= 0
\end{align*}

Therefore, $y_1(t)$ is a solution of the given differential equation. Next we will verify that $y_2(t)$ is a solution
\begin{align*}
    y_2(t) &= e^{-2t} \\
    y_2' &= -2e^{-2t} \\
    y_2'' &= 4e^{-2t} \\
    4e^{-2t} + 2e^{-2t} - 6e^{-2t} &= 0
\end{align*}

Therefore, $y_2(t)$ is a solution of the given differential equation. Next we will show that $c_1y_1(t) + c_2y_2(t)$ is also a solution for any real numbers $c_1$ and $c_2$.

\begin{align*}
    c_1y_1(t) + c_2y_2(t) &= c_1e^{3t} + c_2e^{-2t} \\
\end{align*}

\section*{Problem 6}
\textbf{Problem Statement:} Find the general solution to the non homogenoues ODE $y' + \frac{1}{t+1}y = 2$

\textbf{Solution:} We will solve using the superposition principal. $y_{G,NH} = y_{G,H} + y_{P,NH}$
\subsection*{(a)}
\textbf{Problem Satement:} Show that $y_p = \frac{t^2 + 2t}{t+1}$ is a particular solution to the Non-Homogeneous ODE $y' + \frac{1}{t+1}y = 2$

\textbf{Solution:} We will start by finding the derivative of $y_p$
\begin{align*}
y_p &= \frac{t^2 + 2t}{t+1} \\
y_p' &= \frac{(t+1)(2t+2) - (t^2 + 2t)}{(t+1)^2} \\
y_p' &= \frac{2t^2 + 2t + 2t + 2 - t^2 - 2t}{(t+1)^2} \\
y_p' &= \frac{t^2 + 4t + 2}{(t+1)^2}
\end{align*}

Now we will plug $y_p$ and $y_p'$ into the Non-Homogeneous ODE
\begin{align*}
y_p' + \frac{1}{t+1}y_p &= 2 \\
\frac{t^2 + 4t + 2}{(t+1)^2} + \frac{t^2 + 2t}{(t+1)^2} &= 2 \\
\frac{t^2 + 4t + 2 }{t^2 + 2t + 1} + \frac{t^2+2}{t^2 + 2t + 1} &= 2 \\
\frac{2t^2 + 4t + 2}{t^2 + 2t + 1} &= 2 \\
\frac{2(t^2 + 2t + 1)}{t^2 + 2t + 1} &= 2 \\
2 &= 2
\end{align*}

\subsection*{(b)}
\textbf{Problem Statement:} Find the general solution to the Non-Homogeneous ODE $y' + \frac{1}{t+1}y = 2$

\textbf{Solution:} We will use the superposition principal to find the general solution to the Non-Homogeneous ODE
\begin{align*}
y_{G,NH} &= y_{G,H} + y_{P,NH} \\
\end{align*}

We will start by finding the general solution to the Homogeneous ODE
\begin{align*}
y' + \frac{1}{t+1}y &= 0 \\
\frac{1}{y}dy = -\frac{1}{t+1}dt \\
\ln|y| = -\ln|t+1| + c \\
y = \frac{c}{t+1}
\end{align*}

Now we will find the general solution to the Non-Homogeneous ODE
\begin{align*}
y_{G,NH} &= \frac{c}{t+1} + \frac{t^2 + 2t}{t+1} \\
\end{align*}

\section*{Problem 7}

\section*{Problem 8}
\textbf{Problem Statement:} Solve the differential equation $(x+1)\frac{dy}{x}+y = \ln(t)$

\textbf{Solution:} We will solve using the integrating factor method
We will start by rewriting the equation
\[
\frac{dy}{dt} + \frac{y}{t+1} = \frac{lnt}{t+1}
\]

Set right side equal to zero
\[
\frac{dy}{dt} + \frac{y}{t+1} = 0
\]

Solve for y
\begin{align*}
\frac{dy}{dt} &= -\frac{y}{t+1} \\
\frac{dy}{y} &= -\frac{dt}{t+1} \\
\end{align*}

Integrate both sides
\begin{align*}
\int \frac{1}{y} dy &= -\frac{1}{t+1} dt \\
\ln|y| &= -ln|t+1| + c \\
y &= e^{-\ln|t+1| + c} \\
y&= \frac{c}{t+1}
\end{align*}

Solve for integrating factor $\mu$
\begin{align*}
\mu &= \frac{1}{y_{G,H}} \\
\mu &= t+1
\end{align*}


Multiply orignial equation by $\mu$
\begin{align*}
\mu \frac{dy}{dt} + \mu \frac{y}{t+1} &= \mu \frac{lnt}{t+1} \\
D[(t+1)y] &= \frac{\ln(t)}{t+1}(t+1) \\
(t+1)y &= \int \ln(t) \\
(t+1)y &= t\ln(t) - t + c \\
y &= \frac{t\ln(t) - t + c}{t+1}
\end{align*}

\end{document}