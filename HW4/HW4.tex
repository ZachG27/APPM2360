\documentclass[12pt, letterpaper]{article}
\usepackage[utf8]{inputenc}
\usepackage{amsmath, amsfonts, amssymb}
\usepackage{geometry}
\usepackage{graphicx}
\usepackage{fancyhdr}
\usepackage{pgfplots}
\usepackage{float}
\usepackage{pgfplotstable}
\usepackage{booktabs} 
\usepackage{amsmath} % Add this line to include the amsmath package
\pgfplotsset{compat=1.18}

% Setup for homework style
\geometry{margin=1in}
\pagestyle{fancy}
\fancyhf{}
\lhead{APPM 2360: Differential Equations}
\rhead{Homework 4}
\cfoot{\thepage}

\title{Homework 4}
\author{Zachariah Galdston}
\date{9/19/2024}

\begin{document}

\maketitle

\section*{Problem 1}
\textbf{Problem Statement:} 
Suppose you have $A_0$ dollars to invest in a savings account earning an annual interest rate of $r$ percent compounded continuously. Furthermore, suppose that you make annual deposits of $d$ dollars to the account. The differential equation governing this situation is:

\[
\frac{dA}{dt} = rA + d, \quad A(0) = A_0
\]

\subsection*{(a)}
\textbf{Problem Statement:}
Find an equation for the future value $A(t)$ of the account by solving the initial value problem. 

\textbf{Solution:}
We start with the differential equation:
\[
\frac{dA}{dt} = rA + d
\]

Using the integrating factor method with $\mu = e^{rt}$:
\begin{align*}
\frac{dA}{dt} - rA &= d \\
e^{rt} \frac{dA}{dt} - re^{rt}A &= de^{rt} \\
\frac{d}{dt}(Ae^{rt}) &= de^{rt} \\
\end{align*}

Now integrate both sides:
\[
Ae^{rt} = \int de^{rt} dt = \frac{d}{r}e^{rt} + C
\]
Thus, the general solution is:
\[
A(t) = \frac{d}{r} + Ce^{-rt}
\]
Using the initial condition $A(0) = A_0$, we solve for $C$:
\[
A(0) = \frac{d}{r} + C = A_0 \quad \Rightarrow \quad C = A_0 - \frac{d}{r}
\]
Therefore, the solution is:
\[
A(t) = \frac{d}{r} + \left(A_0 - \frac{d}{r}\right)e^{-rt}
\]

\subsection*{(b)}
\textbf{Problem Statement:}
Upon graduating from college and starting your career, you have no money. You deposit $d = 1000$ into an account that pays interest at a rate of $8\%$ compounded continuously. Find the future value of the account after 40 years and the interest earned.

\textbf{Solution:}
We use the formula from part (a) with $r = 0.08$, $d = 1000$, and $A_0 = 0$. The solution becomes:
\[
A(t) = \frac{1000}{0.08} + \left(0 - \frac{1000}{0.08}\right)e^{-0.08t}
\]
Simplifying:
\[
A(t) = 12500(1 - e^{-0.08t})
\]
Substituting $t = 40$ years:
\[
A(40) = 12500(1 - e^{-0.08 \times 40}) \approx 294156.63
\]
The interest earned is the difference between the final balance and the total deposits:
\[
\text{Interest} = 294156.63 - 1000 \times 40 = 254156.63
\]

\subsection*{(c)}
\textbf{Problem Statement:}
Find the value of $d$ that would produce a balance of one million dollars after 40 years.

\textbf{Solution:}
We need to solve for $d$ in the equation:
\[
1000000 = \frac{d}{0.08}(1 - e^{-0.08 \times 40})
\]
Solving for $d$:
\[
d = \frac{1000000 \times 0.08}{1 - e^{-0.08 \times 40}} \approx 3395.95
\]

\subsection*{(d)}
\textbf{Problem Statement:}
If the annual deposit is $d = 2500$, find the value of $r$ that would produce a balance of one million dollars after 40 years.

\textbf{Solution:}
We need to solve the equation:
\[
1000000 = \frac{2500}{r}(1 - e^{-r \times 40})
\]
This requires numerical methods, and solving yields $r \approx 0.0904$ or $9.04\%$.

\section*{Problem 2}
\textbf{Problem Statement:}
You must be anesthetized with a minimum concentration of 50 milligrams per kilogram. Your weight is 100 kg, and the half-life of the anesthetic is 10 hours. Find the dose needed to stay anesthetized for 3 hours.

\textbf{Solution:}
The elimination of the anesthetic follows an exponential decay model:
\[
A(t) = A_0 e^{-kt}
\]

We will solve for $k$ using the half-life of 10 hours:

\begin{align*}
\frac{1}{2}A_0 &= A_0 e^{-10k} \\
\frac{1}{2} &= e^{-10k} \\
\ln \frac{1}{2} &= -10k \\
k &= \frac{\ln 2}{10}
\end{align*}
At $t = 3$ hours, we need $A(3) = 50 \times 100 = 5000$ mg. Solving for $A_0$:
\[
5000 = A_0 e^{-3 \frac{\ln 2}{10}} \quad \Rightarrow \quad A_0 \approx 6160 \text{ mg or 6.16 grams}
\]

\section*{Problem 3}
\textbf{Problem Statement:}
A tank starts with 300 gallons of pure water, and a salt solution flows in at 3 gallons per minute while solution drains at 1 gallon per minute. Find the salt content when the tank reaches 600 gallons.

\textbf{Solution:}
The differential equation governing the salt content $S(t)$ is:
\begin{align*}
\frac{dS}{dt} &= (RateIn)(ConcentrationIn) - (RateOut)(ConcentrationOut) \\
&= (3 \cdot 1) - (\frac{S}{300+2t} \cdot 1) 
\end{align*}
Solving the differential equation:
\begin{align*}
\frac{dS}{dt} &= 3 - \frac{S}{300+2t} \\
\frac{dS}{dt} + \frac{S}{300+2t} &= 3
\end{align*}

We use the integrating factor method with $\mu = e^{\int \frac{1}{300+2t} dt} = e^{\frac{1}{2} \ln(300+2t)} = (300+2t)^{\frac{1}{2}}$:
\begin{align*}
(300+2t)^{\frac{1}{2}} \frac{dS}{dt} + (300+2t)^{-\frac{1}{2}}S &= 3 \\
\frac{d}{dt}((300+2t)^{\frac{1}{2}}S) &= 3(300+2t)^{\frac{1}{2}} \\
\end{align*}

Integrating both sides:
\begin{align*}
(300+2t)^{\frac{1}{2}}S &= 3 \int (300+2t)^{\frac{1}{2}} dt \\
\end{align*}

Solving the right side integral:
\begin{align*}
3 \int (300+2t)^{\frac{1}{2}} dt &= 3 \left( \frac{2}{3} (300+2t)^{\frac{3}{2}} \right) + C \\
&= (300+2t)^{\frac{3}{2}} + C
\end{align*}

Substitute back into the differential equation:

\begin{align*}
(300+2t)^{\frac{1}{2}}S &= (300+2t)^{\frac{3}{2}} + C \\
S &= (300+2t) + \frac{C}{(300+2t)^{-\frac{1}{2}}}
\end{align*}

Solving for $C$ using the initial condition $S(0) = 0$:

\begin{align*}
0 &= 300 + \frac{C}{300^{-\frac{1}{2}}} \\
C = -300 \cdot 300^\frac{1}{2} 
\end{align*}

Therefore, the solution is:

\[ 
S(t) = 300 + 2t - \frac{300(300^\frac{1}{2})}{(300+2t)^{-\frac{1}{2}}}
\]

To solve for the time when the tank reaches 600 gallons, we set must solve for $t$. We know the tank is increaseing at rate $RateIN - RateOut = 3-1$ gallons per minute, so we solve for $t$ in the equation $600 = 300 + 2t$ to find $t = 150$ minutes. Substituting $t = 150$ into the solution:

\begin{align*}
S(t) &= 300 + 2(t) - \frac{300(300^\frac{1}{2})}{(300+2t)^{-\frac{1}{2}}} \\
S(150) &= 300 + 2(150) - \frac{300(300^\frac{1}{2})}{(300+2(150))^{-\frac{1}{2}}} \\
&= 600 - \frac{300(300^\frac{1}{2})}{(600)^{-\frac{1}{2}}} \\
&= 600 - \frac{300(300^\frac{1}{2})}{\sqrt{600}} \\
&\approx 387.87 lbs
\end{align*}
    
\section*{Problem 4}
\textbf{Problem Statement:}
A small single-stage rocket of mass $m(t)$ is launched vertically. The air resistance is linear, and the rocket consumes fuel at a constant rate. The velocity of the rocket is modeled by the differential equation:
\[
\frac{dv}{dt} + \frac{k - \lambda}{m_0 - \lambda t} v = -g + \frac{R}{m_0 - \lambda t}
\]
where $m_0 = 200$ kg, $R = 2000$ N, $\lambda = 1$ kg/s, $g = 9.8$ m/s\(^2\), $k = 3$ kg/s, and $v(0) = 0$.

\subsection*{(a)}
\textbf{Problem Statement:}
Find the velocity $v(t)$ of the rocket.

\textbf{Solution:}
Solve using the integrating factor method with $\mu = (200-t)^{-2}$:
\begin{align*}
\frac{dv}{dt} + \frac{3-1}{200-t}v = -9.8 + \frac{2000}{200-t} \\
(200-t)^{-2} \frac{dv}{dt} + (200-t)^{-3}v = -9.8(200-t)^{-2} + 2000(200-t)^{-3} \\
\frac{d}{dt}((200-t)^{-2}v) = -9.8(200-t)^{-2} + 2000(200-t)^{-3} \\
\end{align*}

Integrating both sides:

\begin{align*}
(200-t)^{-2}v &= 9.8(200-t)^{-1} - 1000(200-t)^{-2} + C \\
v &= 9.8(200-t) - 1000 + C(200-t)^2
\end{align*}

Solving for $C$ using the initial condition $v(0) = 0$:

\begin{align*}
0 &= 9.8(200) - 1000 + C(200)^2 \\
C &= \frac{9.8(200)-1000}{200^2} = \frac{1960-1000}{40000} = 0.024
\end{align*}

So the final solution is:

\[
v(t) = 9.8(200-t) - 1000 + 0.024(200-t)^2
\]

\subsection*{(b)}
\textbf{Problem Statement:}
Find the height $s(t)$ of the rocket.

\textbf{Solution:}
The height is found by integrating $v(t)$:
\[
s(t) = \int v(t) dt
\]

Substitute $v(t)$ into the integral and C = 0:

\begin{align*}
s(t) &= \int -9.8(200-t) + 1000 + 0.024(200-t)^2 dt \\
&= \int 0.024t^2 + 0.2t dt \\
&= 0.008t^3 + 0.1t^2 + C
&= 0.008t^3 + 0.1t^2
\end{align*}


\subsection*{(c)}
\textbf{Problem Statement:}
Find the burnout time $t_b$ when all the fuel is consumed.

\textbf{Solution:}
The burnout time is when $m(t) = 200 - \lambda t = 200-50$. Solving for $t$:
\begin{align*}
200 - 50 &= 200 - t \\ 
t &= 50 \text{ seconds}
\end{align*}


\subsection*{(d)}
\textbf{Problem Statement:}
Find the velocity at burnout.

\textbf{Solution:}
Substitute $t_b = 50$ s into the velocity equation $v(t)$:
\[
v(50) = 9.8(200-50) - 1000 + 0.024(200-50)^2
\]
After simplifying, the velocity at burnout is approximately $v(50) \approx 1010 \text{ m/s}$.

\subsection*{(e)}
\textbf{Problem Statement:}
Find the height at burnout.

\textbf{Solution:}
Substitute $t_b = 50$ s into $s(t)$ to find the height:
\[
s(50) = 0.008(50)^3 + 0.1(50)^2 = 1250 \text{ m}
\]
The height at burnout is approximately $s(50) \approx 1250 \text{ m}$.

\section*{Problem 5}
\textbf{Problem Statement:}
Your air conditioner breaks down at noon. The temperature inside the house is 75$^\circ$F, and outside it is 95$^\circ$F. The time constant for the house is 4 hours. Find the temperature inside the house after 2 hours.

\subsection*{(a)}
\textbf{Solution:}
The temperature follows Newton's Law of Cooling:
\[
T(t) = T_{\text{out}} + (T_0 - T_{\text{out}}) e^{-kt}
\]
where $k = \frac{1}{4}$ and $T_0 = 75^\circ$F. Substituting $t = 2$ hours:
\[
T(2) = 95 + (75 - 95)e^{-\frac{2}{4}} \approx 82.87^\circ F
\]

\subsection*{(b)}
\textbf{Problem Statement:}
Find the time when the temperature reaches 80$^\circ$F.

\textbf{Solution:}
We solve for $t$ in the equation:
\[
80 = 95 + (75 - 95)e^{-\frac{t}{4}}
\]
Solving this gives:
\[
t \approx 1 \text{ hour and } 9 \text{ minutes}
\]

\section*{Problem 6}
\textbf{Problem Statement:}
Consider the differential equation for population growth:
\[
\frac{dP}{dt} = kP^{1+c}
\]
where $c = 0.01$. Find the solution given the initial population and doubling rate.

\subsection*{(a)}
\textbf{Solution:}
We separate variables and integrate:
\begin{align*}
    \int P^{-1.01} dP &= \int k dt \\
    \frac{-100}{P^\frac{1}{100}} &= kt + C
\end{align*}

Solving for C with the initial condition $P(0) = 10$:

\begin{align*}
    \frac{-100}{10^\frac{1}{100}} &= 0 + C \\
    C &= -97.7237
\end{align*}

Solving for $k$ with the doubling rate condition $P(5) = 20$:

\begin{align*}
    \frac{-100}{20^\frac{1}{100}} &= 5k - 97.7237 \\
    \frac{-100}{(20)^\frac{1}{100}} + 97.7237 &= 5k \\
    k &\approx 0.135
\end{align*}

Therefore, the solution is:

\[
P(t) =  \left( \frac{0.135t - 97.7237}{-100} \right)^{-100}
\]
\subsection*{(b)}
\textbf{Problem Statement:}
Find the population after 50 and 100 months.

\textbf{Solution:}
Substitute $t = 50$ and $t = 100$ into the solution from part (a):

\begin{align*}
    P(50) &=  \left( \frac{0.135(50) - 97.7237}{-100} \right)^{-100} \approx 12,835
    P(100) &= \left( \frac{0.135(100) - 97.7237}{-100} \right)^{-100} \approx 28,613,327
\end{align*}

\subsection*{(c)}
\textbf{Problem Statement:}
Find the doomsday time $t_0$ when the population becomes infinite.

\textbf{Solution:}
The population becomes infinite when there is a 0 in the demoninator of the solution from part (a). Solving for $t$:

\begin{align*}
    \frac{-100}{P^\frac{1}{100}} &= kt + C
    \frac{-100}{kt + c} = P^\frac{1}{100}
    \frac{-100}{0.135t - 97.7237} = P^\frac{1}{100}
\end{align*}

Sett8ng the denominator to 0 gives:

\begin{align*}
    0.135t - 97.7237 &= 0 \\
    t &= \frac{97.7237}{0.135} \approx 724.28 \text{ months}
\end{align*}

\end{document}
